% Options for packages loaded elsewhere
\PassOptionsToPackage{unicode}{hyperref}
\PassOptionsToPackage{hyphens}{url}
%
\documentclass[
]{article}
\author{}
\date{}

\usepackage{amsmath,amssymb}
\usepackage{lmodern}
\usepackage{iftex}

\usepackage[T1]{fontenc}
\usepackage{lmodern}

\begin{document}

\hypertarget{api-information-rsa-codesign}{%
\section{API information RSA
codesign}\label{api-information-rsa-codesign}}

\begin{itemize}
\tightlist
\item
  \protect\hyperlink{api-information-rsa-codesign}{API information RSA
  codesign}

  \begin{itemize}
  \tightlist
  \item
    \protect\hyperlink{input-register}{Input register}

    \begin{itemize}
    \tightlist
    \item
      \protect\hyperlink{reasoning-and-discussion}{Reasoning and
      discussion}
    \end{itemize}
  \item
    \protect\hyperlink{output-register}{Output register}
  \end{itemize}
\end{itemize}

\hypertarget{input-register}{%
\subsection{Input register}\label{input-register}}

\begin{longtable}[]{@{}ccc@{}}
\toprule
Register & Description & Explanations \\
\midrule
\endhead
Rin0 & command register & Main register to send input command \\
Rin1 & dma\_rx\_address & here is the DMA receive address \\
Rin2 & dma\_tx\_address & here is the DMA transmit address \\
Rin3 & t & saves the value of the exponent \\
Rin4 & t\_len & used during the loading of the data \\
Rin5 & Loading data status & Command to indicate the state of the
loading \\
Rin6 & & \\
Rin7 & & \\
\bottomrule
\end{longtable}

\hypertarget{reasoning-and-discussion}{%
\subsubsection{Reasoning and
discussion}\label{reasoning-and-discussion}}

I have decided to not transfer the exponet e over DMA since most common
choice of exponents for RSA algorithm is a 16 bits + 1 integers and all
of the vector test that I produced gave me some 16 bits exponents.
Theoritically, we could go higher but we will loose in speed for any RSA
implementation. So since we are taking less than 32 bits, I will
transfer e and its length through a register and not wasting clock
cycles loading it.

We will only 3 loading operations to load N,R\_N and R2\_N.

\hypertarget{output-register}{%
\subsection{Output register}\label{output-register}}

\begin{longtable}[]{@{}ccc@{}}
\toprule
Register & Description & Explanations \\
\midrule
\endhead
Rout0 & & \\
Rout1 & & \\
Rout2 & & \\
Rout3 & & \\
Rout4 & & \\
Rout5 & & \\
Rout6 & & \\
Rout7 & & \\
\bottomrule
\end{longtable}

\end{document}
