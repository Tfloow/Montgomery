\hypertarget{api-information-rsa-codesign}{%
\section{API information RSA
codesign}\label{api-information-rsa-codesign}}

\begin{itemize}
\tightlist
\item
  \protect\hyperlink{api-information-rsa-codesign}{API information RSA
  codesign}

  \begin{itemize}
  \tightlist
  \item
    \protect\hyperlink{input-register}{Input register}

    \begin{itemize}
    \tightlist
    \item
      \protect\hyperlink{reasoning-and-discussion}{Reasoning and
      discussion}
    \end{itemize}
  \item
    \protect\hyperlink{output-register}{Output register}
  \end{itemize}
\end{itemize}

\hypertarget{input-register}{%
\subsection{Input register}\label{input-register}}

\begin{longtable}[]{@{}
  >{\centering\arraybackslash}p{(\columnwidth - 4\tabcolsep) * \real{0.1127}}
  >{\centering\arraybackslash}p{(\columnwidth - 4\tabcolsep) * \real{0.2676}}
  >{\centering\arraybackslash}p{(\columnwidth - 4\tabcolsep) * \real{0.6197}}@{}}
\toprule\noalign{}
\begin{minipage}[b]{\linewidth}\centering
Register
\end{minipage} & \begin{minipage}[b]{\linewidth}\centering
Description
\end{minipage} & \begin{minipage}[b]{\linewidth}\centering
Explanations
\end{minipage} \\
\midrule\noalign{}
\endhead
\bottomrule\noalign{}
\endlastfoot
Rin0 & command register & Main register to send input command \\
Rin1 & dma\_rx\_address & here is the DMA receive address \\
Rin2 & dma\_tx\_address & here is the DMA transmit address \\
Rin3 & t & saves the value of the exponent \\
Rin4 & t\_len & used during the loading of the data \\
Rin5 & Loading data status & Command to indicate the state of the
loading \\
Rin6 & & \\
Rin7 & & \\
\end{longtable}

\hypertarget{reasoning-and-discussion}{%
\subsubsection{Reasoning and
discussion}\label{reasoning-and-discussion}}

I have decided to not transfer the exponet e over DMA since most common
choice of exponents for RSA algorithm is a 16 bits + 1 integers and all
of the vector test that I produced gave me some 16 bits exponents.
Theoritically, we could go higher but we will loose in speed for any RSA
implementation. So since we are taking less than 32 bits, I will
transfer e and its length through a register and not wasting clock
cycles loading it.

We will only 3 loading operations to load N,R\_N and R2\_N.

\hypertarget{output-register}{%
\subsection{Output register}\label{output-register}}

\begin{longtable}[]{@{}
  >{\centering\arraybackslash}p{(\columnwidth - 4\tabcolsep) * \real{0.1143}}
  >{\centering\arraybackslash}p{(\columnwidth - 4\tabcolsep) * \real{0.2000}}
  >{\centering\arraybackslash}p{(\columnwidth - 4\tabcolsep) * \real{0.6857}}@{}}
\toprule\noalign{}
\begin{minipage}[b]{\linewidth}\centering
Register
\end{minipage} & \begin{minipage}[b]{\linewidth}\centering
Description
\end{minipage} & \begin{minipage}[b]{\linewidth}\centering
Explanations
\end{minipage} \\
\midrule\noalign{}
\endhead
\bottomrule\noalign{}
\endlastfoot
Rout0 & Status & indicate the status of the Hardware \\
Rout1 & LSB\_N & will write the 32 last bits of the register N \\
Rout2 & LSB\_R\_N & will write the 32 last bits of the register R\_N \\
Rout3 & LSB\_R2\_N & will write the 32 last bits of the register
R2\_N \\
Rout4 & dma\_rx\_address & dbg to check the address of the DMA
receive \\
Rout5 & Loading & indicate what data are we loading \\
Rout6 & State & indicate the state of the FSM \\
Rout7 & & \\
\end{longtable}
